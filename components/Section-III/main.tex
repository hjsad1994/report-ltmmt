
\newpage

\section{CHƯƠNG 1: TỔNG QUAN VỀ NICEPHIM }

\subsection{Giới thiệu}

NicePhim. là một nền tảng mang đến trải nghiệm xem phim trực tuyến, được thiết kế nhằm giúp người dùng và những người thân yêu có thể cùng nhau thưởng thức các tác phẩm điện ảnh, xóa bỏ mọi rào cản về địa lý. Nền tảng được phát triển dựa trên hai tính năng cốt lõi:

\begin{itemize}
	\item \textbf{Xem Phim Chung (Watch Party):} Tính năng cho phép người dùng tạo các phòng chiếu phim riêng tư, nơi video được đồng bộ hóa một cách hoàn hảo cho tất cả thành viên. Điều này cho phép mọi người có thể cùng theo dõi và bình luận về một bộ phim dù ở bất kỳ đâu.

	\item \textbf{Trò Chuyện Trực Tiếp (Live Chat):} Khung trò chuyện được tích hợp trực tiếp trong phòng chiếu, cho phép các thành viên chia sẻ cảm xúc và bình luận tức thì trong suốt quá trình xem phim. Tính năng này làm tăng tính tương tác và tạo cảm giác như đang cùng xem tại một địa điểm.
\end{itemize}

Mục tiêu của NicePhim là biến mỗi buổi xem phim thành một sự kiện kết nối trực tuyến, giúp gắn kết bạn bè và tạo ra những kỷ niệm đáng nhớ.

% --- Tổng quan hệ thống ---

\subsection{Tổng quan hệ thống}

NicePhim được xây dựng với kiến trúc hiện đại, gồm ba thành phần chính:
\begin{itemize}
	\item \textbf{Frontend:} Sử dụng Next.js 15, React 19, TypeScript và Tailwind CSS, đảm bảo giao diện hiện đại, responsive, trải nghiệm người dùng mượt mà trên mọi thiết bị.
	\item \textbf{Backend:} Phát triển bằng Spring Boot 3.1.5 (Java 17), cung cấp API, xử lý logic nghiệp vụ, xác thực người dùng, quản lý phim, thể loại, và đồng bộ dữ liệu thời gian thực qua WebSocket (STOMP).
	\item \textbf{Database:} Lưu trữ dữ liệu trên Microsoft SQL Server, sử dụng Flyway để quản lý migration, đảm bảo tính nhất quán và mở rộng.
\end{itemize}

Hệ thống hỗ trợ streaming video chất lượng cao (360p, 480p, 720p, 1080p, 4K) với phụ đề, quản lý tập phim, lịch sử xem, yêu thích, và các tính năng xã hội như bình luận, đánh giá. Tính năng “Xem chung” sử dụng WebSocket để đồng bộ phát video giữa các thành viên trong phòng.

% --- Thiết kế tương tác ---

\subsection{Thiết kế tương tác}

Giao diện NicePhim được thiết kế theo hướng hiện đại, lấy người dùng làm trung tâm:
\begin{itemize}
	\item \textbf{Đăng ký/Đăng nhập:} Quy trình đơn giản, xác thực qua email hoặc tên người dùng, bảo mật bằng BCrypt.
	\item \textbf{Khám phá nội dung:} Người dùng có thể tìm kiếm, lọc phim theo thể loại, năm, đánh giá, hoặc nhận đề xuất cá nhân hóa.
	\item \textbf{Xem phim:} Trình phát video hỗ trợ chọn chất lượng, phụ đề, tua nhanh/chậm, toàn màn hình, và ghi nhớ tiến trình xem.
	\item \textbf{Xem chung:} Tạo/join phòng, đồng bộ phát video, trò chuyện trực tiếp, quản lý thành viên phòng.
	\item \textbf{Quản trị viên:} Giao diện riêng để quản lý phim, thể loại, kiểm duyệt nội dung.
\end{itemize}

Thiết kế hướng tới trải nghiệm liền mạch, tối ưu thao tác, hỗ trợ tốt trên cả máy tính và thiết bị di động.

% --- Phương pháp tiếp cận và giải quyết vấn đề ---

\subsection{Phương pháp tiếp cận và giải quyết vấn đề}

Dự án áp dụng các phương pháp và công nghệ hiện đại để giải quyết các thách thức:
\begin{itemize}
	\item \textbf{Streaming mượt mà:} Sử dụng FFmpeg chuyển đổi video sang HLS đa chất lượng, tích hợp HLS.js trên frontend để tự động chọn chất lượng phù hợp.
	\item \textbf{Đồng bộ thời gian thực:} WebSocket (STOMP) đảm bảo mọi thành viên trong phòng xem chung luôn đồng bộ trạng thái phát video.
	\item \textbf{Bảo mật:} Xác thực người dùng với BCrypt, kiểm soát truy cập API, chống SQL injection, xác thực đầu vào.
	\item \textbf{Quản lý dữ liệu:} Sử dụng UUID cho khóa chính, migration Flyway, thiết kế database chuẩn hóa, hỗ trợ mở rộng.
	\item \textbf{Tối ưu trải nghiệm:} Giao diện responsive, tối ưu hiệu năng, phân biệt cảnh báo/lỗi rõ ràng, hỗ trợ đa nền tảng.
\end{itemize}

% --- Tổng kết chương ---

\subsection{Tổng kết chương}

Chương này đã trình bày tổng quan về nền tảng NicePhim, bao gồm mục tiêu, kiến trúc hệ thống, thiết kế tương tác và các phương pháp tiếp cận kỹ thuật. Những nền tảng này là cơ sở để phát triển các chức năng chi tiết ở các chương tiếp theo.
